% Baseado em
% https://www.overleaf.com/latex/templates/cse-3500-algorithms-and-complexity-homework-template/wrfwdhfzpnqc
\documentclass[12pt,letterpaper]{article}
\usepackage{fullpage}
\usepackage[top=2cm, bottom=4.5cm, left=2.5cm, right=2.5cm]{geometry}
\usepackage{amsmath,amsthm,amsfonts,amssymb,amscd}
\usepackage{lastpage}
\usepackage{enumerate}
\usepackage{fancyhdr}
\usepackage{mathrsfs}
\usepackage{xcolor}
\usepackage{graphicx}
\usepackage{listings}
\usepackage{hyperref}
\usepackage{multicol}
\usepackage{xspace}

\usepackage[brazilian]{babel}

\hypersetup{%
  colorlinks=true,
  linkcolor=blue,
  linkbordercolor={0 0 1}
}

\setlength{\parindent}{0.0in}
\setlength{\parskip}{0.05in}

% Edit these as appropriate
\newcommand\course{MAC0336}
\newcommand\prof{Routo Terada}
\newcommand\hwnumber{1}                   % <-- homework number
\newcommand\NUSP{Seu NUSP}                % <-- NUSP
\newcommand\sname{Seu nome}               % <-- Name

\newcommand\answer{\textbf{Resolução.}\xspace}

\pagestyle{fancyplain}
\headheight 35pt
\lhead{\sname \\ \NUSP}
\chead{\textbf{\Large Lista \hwnumber}}
\rhead{\course\, - \prof \\ \today}       % \today deixa o dia de hoje automaticamente
\lfoot{}
\cfoot{}
\rfoot{\center\small\thepage}
\headsep 1.5em

\begin{document}

% Use \section*{} em vez de \section{} para evitar que o Latex numere as
% seções. Isso evita que fique redundante "1 Exercício 1", por exemplo.
\section*{Exercício 1}

Uma fonte de informação $\mathcal{X}$ gera saídas $\{ x_1, \ldots, x_n\}$
com as probabilidades $p(x_1),$ $\ldots,$ e $p(x_n)$.
Lembre que a entropia de Shannon é definida como:
\[
    H \left( \mathcal{X} \right) = \sum_{i = 1}^{n} p(x_i) \log_2 \left( \frac{1}{p(x_i)} \right).
\]

%%%%%%%% ITEM 1.1 %%%%%%%%
\subsection*{1.1}
Escrever todos os passos do cálculo da entropia de $X$ para as seguintes probabilidades:
% Usei multicols aqui para os itens ocuparem menos espaço.
\begin{multicols}{3} % Poderia tirar essa linha
  \begin{itemize}
    \item $p(x_1) = 1/4$,
    \item $p(x_2) = 1/16$,
    \item $p(x_3) = 1/16$,
    \item $p(x_4) = 1/16$,
    \item $p(x_5) = 1/4$,
    \item $p(x_6) = 1/16$,
    \item $p(x_7) = 1/4$.
    \item[\vspace{\fill}] % Placeholder para ocupar as linhas que faltam quando
                          % os itens 7 são quebrados em colunas de 3.
  \end{itemize}
\end{multicols} % Poderia tirar essa linha

%%%%%%%% Resposta do item 1.1 %%%%%%%%

\answer
Usando a definição de entropia e aplicando os valores das probabilidades acima,
temos \ldots

\qed % Desenha um quadradinho no fim para explicitar que acabou a resolução.
%%%%%%%% FIM DO ITEM 1.1 %%%%%%%%


\subsection*{1.2}
\answer

\qed

\subsection*{1.3}
\answer

\qed

\subsection*{1.4}

\qed

\answer

\section*{Exercício 2}

\end{document}